\documentclass{beamer}
\usepackage{alltt}
\usepackage{tikz}
\usetikzlibrary{matrix}
\usetikzlibrary{trees}
\usepackage{cancel}
\usepackage{subcaption}
\PassOptionsToPackage{obeyspaces}{url}
\usepackage{hyperref}

\usepackage{lipsum}

\usetheme{Singapore}
\definecolor{violet}{RGB}{205,99,170}
\setbeamercolor{title}{fg=violet}
\setbeamercolor{frametitle}{fg=violet}
\setbeamercolor{structure}{fg=violet}

\newcommand{\racket}{\texttt{Racket}}
\newcommand{\drr}{\texttt{DrRacket}}
\newcommand{\fsm}{\texttt{FSM}}
\newcommand{\ide}{\texttt{IDE}}
\newcommand{\api}{\texttt{API}}
\newcommand{\arrow}{\(\rightarrow\)}
\newcommand{\dotss}{\(\ldots\)}
\newcommand{\vdotss}{\(\vdots\)}
\newcommand{\elist}{\texttt{\textquotesingle{()}}}
\newcommand{\logand}{\texttt{\(\wedge\)}}
\newcommand{\logor}{\texttt{\(\vee\)}}
\newcommand{\imp}{\texttt{\(\Rightarrow\)}}
\newcommand{\sig}{\texttt{\(\Sigma\)}}
\newcommand{\delt}{\texttt{\(\delta\)}}
\newcommand{\sigsig}{\texttt{\(\Sigma\) = \{a b\}}}
\newcommand{\gam}{\texttt{\(\Gamma\)}}
\newcommand{\ep}{\texttt{\(\epsilon\)}}
\newcommand{\quot}{\texttt{\textquotesingle{}}}
\newcommand{\dquot}{\texttt{"}}
\newcommand{\qquot}{\texttt{\textasciigrave{}}}
\newcommand{\lambexpr}{\texttt{$\lambda$}-expression}
\newcommand{\lamb}{\texttt{$\lambda$}}
\newcommand{\is}{\texttt{::=}}
\newcommand{\dfa}{\texttt{dfa}}
\newcommand{\ndfa}{\texttt{ndfa}}



\begin{document}

\title{Project 1}
%\subtitle{Using Beamer}
\author{Luca Guerini and Tijana Minic}
\institute{Seton Hall University}
\date{}

\begin{frame}
\titlepage
\end{frame}

\section{Introduction}

\begin{frame}[fragile]
\frametitle{Introduction}
\begin{itemize}
\item<1-> \texttt{Components}:
\item<2-> Registers: AF, BC, DE, HL, PC, SP
\item<3-> Instruction Execution: Decoding and execution of Z80 instructions
\end{itemize}

\end{frame}

\begin{frame}[fragile]
\frametitle{Registers and Flags}
\begin{itemize}
\item<1-> \texttt{Registers}
\begin{itemize}
\item[\arrow] AF, BC, DE, HL: 16-bit registers for various operations.

\item[\arrow]  PC (Program Counter): Keeps track of the instruction being executed.

\item[\arrow] SP (Stack Pointer): Manages program stack.

\end{itemize}

\item<2->\texttt{Flags}:
\begin{itemize}
\item[\arrow] c (Carry)

\item[\arrow]  h (Half Carry)
\item[\arrow] n (Add/Subtract)
\item[\arrow] z (Zero)

\end{itemize}
\end{itemize}
\end{frame}


\begin{frame}[fragile]
\frametitle{Implementation details}
\begin{itemize}
\item<1-> \texttt{Data Structures:}
\begin{itemize}
\item[\arrow] Registers Class: Implemented using data classes and mutable mapping.

\item[\arrow]  Opcode Parsing: Utilizes \texttt{opcode-parser} for instruction decoding.

\end{itemize}

\item<2->\texttt{Instruction Decoding}:
\begin{itemize}
\item[\arrow] Execute Method: Executes Z80 instructions based on opcode patterns.
\item[\arrow]  Error Handling: Raises InstructionError for unsupported instructions.

\end{itemize}
\end{itemize}
\end{frame}


\begin{frame}[fragile]
\frametitle{IExecution Process}
\begin{itemize}
\item<1-> \texttt{Decode and Execute Loop:}
\begin{itemize}

\item[\arrow] Memory Addressing: Retrieves instructions from memory using program counter.

\item[\arrow]  Decoding: Uses the decoder to interpret opcodes and fetch corresponding instructions.

\item[\arrow]  Execution: Modifies CPU state based on the decoded instruction.

\end{itemize}

\item<2->\texttt{Continual Loop}: Runs indefinitely, simulating continuous instruction execution.
\end{itemize}
\end{frame}


\end{document} 