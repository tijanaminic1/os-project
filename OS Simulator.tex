\documentclass{beamer}
\usepackage{alltt}
\usepackage{tikz}
\usetikzlibrary{matrix}
\usetikzlibrary{trees}
\usepackage{cancel}
\usepackage{subcaption}
\PassOptionsToPackage{obeyspaces}{url}
\usepackage{hyperref}

\usepackage{lipsum}

\usetheme{Singapore}
\definecolor{violet}{RGB}{205,99,170}
\setbeamercolor{title}{fg=violet}
\setbeamercolor{frametitle}{fg=violet}
\setbeamercolor{structure}{fg=violet}

\newcommand{\racket}{\texttt{Racket}}
\newcommand{\drr}{\texttt{DrRacket}}
\newcommand{\fsm}{\texttt{FSM}}
\newcommand{\ide}{\texttt{IDE}}
\newcommand{\api}{\texttt{API}}
\newcommand{\arrow}{\(\rightarrow\)}
\newcommand{\dotss}{\(\ldots\)}
\newcommand{\vdotss}{\(\vdots\)}
\newcommand{\elist}{\texttt{\textquotesingle{()}}}
\newcommand{\logand}{\texttt{\(\wedge\)}}
\newcommand{\logor}{\texttt{\(\vee\)}}
\newcommand{\imp}{\texttt{\(\Rightarrow\)}}
\newcommand{\sig}{\texttt{\(\Sigma\)}}
\newcommand{\delt}{\texttt{\(\delta\)}}
\newcommand{\sigsig}{\texttt{\(\Sigma\) = \{a b\}}}
\newcommand{\gam}{\texttt{\(\Gamma\)}}
\newcommand{\ep}{\texttt{\(\epsilon\)}}
\newcommand{\quot}{\texttt{\textquotesingle{}}}
\newcommand{\dquot}{\texttt{"}}
\newcommand{\qquot}{\texttt{\textasciigrave{}}}
\newcommand{\lambexpr}{\texttt{$\lambda$}-expression}
\newcommand{\lamb}{\texttt{$\lambda$}}
\newcommand{\is}{\texttt{::=}}
\newcommand{\dfa}{\texttt{dfa}}
\newcommand{\ndfa}{\texttt{ndfa}}



\begin{document}

\title{Visualizing a Nondeterministic to Deterministic Finite-State Machine Transformation}
%\subtitle{Using Beamer}
\author{Marco T. Moraz\'{a}n and \emph{Tijana Minic}}
\institute{Seton Hall University}
\date{}

\begin{frame}
\titlepage
\end{frame}

\section{Introduction}

\begin{frame}[fragile]
\frametitle{Introduction}
\begin{scriptsize}
\begin{itemize}
\item<1-> \texttt{FLAT} courses emphasize constructive proofs

\item<1-> Establish the equivalence of different computation models

\item<1-> Our focus today: an \ndfa{} to \dfa{} transformation
\end{itemize}
\end{scriptsize}
\end{frame}

\begin{frame}[fragile]
\frametitle{Introduction}
\begin{scriptsize}
\begin{itemize}
\item<1-> Why do students find this transformation difficult to understand?

\item<2-> The \dfa{} simulates all paths in the \ndfa{}

\begin{itemize}
\scriptsize
\item[\arrow] A path in the constructed \dfa{} may represent multiple paths in the \ndfa{}

\item[\arrow]  A \dfa{} state may represent a subset of states in the \texttt{ndfa}

\item[\arrow] A \dfa{} edge may represent multiple \ndfa{} edges

\end{itemize}

\item<3-> Our tool: visualize addition of \texttt{dfa}-edges while simultaneously visualizing corresponding \texttt{ndfa}-edges


\end{itemize}
\end{scriptsize}
\end{frame}

\section{A Brief Introduction to \fsm}
\subsection{Core Definitions}

\begin{frame}[fragile]
\frametitle{A Brief Introduction to \fsm}
%\framesubtitle{Core Definitions}
\begin{scriptsize}
\begin{itemize}
\normalsize
\item<1-> Machine constructors:
\begin{alltt}
      make-dfa: K \sig{} s F \delt{} [\quot{}no-dead] \arrow{} dfa
     make-ndfa: K \sig{} s F \delt{}            \arrow{} ndfa

     K: list of states
     \sig{}: list of alphabet symbols
     s: start  state in K
     F: list of final states in K
     \delt{}: transition relation
\end{alltt}

\item<2-> Observers:
\begin{alltt}
     (sm-states M) \arrow{} K
     (sm-sigma M)  \arrow{} \sig
     (sm-start M)  \arrow{} s
     (sm-finals M) \arrow{} F
     (sm-rules M)  \arrow{} \delt{}
     (sm-graph M)  \arrow image
\end{alltt}

\end{itemize}
\end{scriptsize}
\end{frame}


\subsection{Illustrative Example}

\begin{frame}[fragile]
\frametitle{A Brief Introduction to \fsm}
%\framesubtitle{Illustrative Example}
\begin{scriptsize}
\begin{itemize}

\item<1-> To illustrate programming in \fsm{}, consider the following example:
\begin{alltt}
;; L(LNDFA) = {\(\epsilon\)} \(\cup\) aa\(\sp{\texttt{*}}\) \(\cup\) ab\(\sp{\texttt{*}}\)
(define LNDFA (make-ndfa \quot{}(S A B F)
                         \quot{}(a b)
                         \quot{}S
                         \quot{}(A B F)
                         \qquot{}((S a A) (S a B)
                           (S ,EMP F) (A b A)
                           (B a B))))
\end{alltt}

\item<2-> \texttt{(sm-graph LNDFA)}
\begin{center}
\includegraphics[scale=0.37]{aa-ab-ndfa-graph.png}
\end{center}

\end{itemize}
\end{scriptsize}
\end{frame}


\begin{frame}[fragile]
\frametitle{Related Work}
\begin{scriptsize}
\begin{itemize}
\item<1-> Most textbooks use the following \ndfa{}\arrow\dfa{} transformation:

\item<2-> \texttt{N = (make-ndfa K \sig{} s F \delt{})} \arrow \texttt{  D = (make-dfa K$^{\prime}$ \sig{}$^{\prime}$ s$^{\prime}$ F$^{\prime}$ \delt{}$^{\prime}$)}

\item<3-> \texttt{D} is constructed as follows in many textbooks:
\begin{description}
  \item [K$^{\prime}$ = ] 2$^K$
  \item[\sig{}$^{\prime}$ =] \sig{}
  \item[s$^{\prime}$ =] E(s)
  \item[F$^{\prime}$ =] \{B $|$ B$\subseteq$K$^{\prime}$ $\wedge$ B$\cap$F$\neq \emptyset$\}
  \item[\delt{}$^{\prime}$ =] \{(Q y $\bigcup$E(p)) $|$ p$\in$K $\wedge$ (q y p)$\in$\delt{} $\wedge$ Q$\in$K$^{\prime} \wedge$ q$\in$Q\} \newline
\end{description}

\item<3-> \texttt{E(s)} = all states reachable from \texttt{s} by empty transitions

\end{itemize}
\end{scriptsize}
\end{frame}


\section{Related Work}

\begin{frame}[fragile]
\frametitle{Related Work}
\begin{large}
\begin{itemize}
\item<1-> JFLAP visualization tool demonstration

\begin{itemize}

\item[\arrow] OpenFLAP is a similar tool with less functionalities, but the haphazard transition diagrams persist as the main problem.

\end{itemize}

\end{itemize}
\end{large}
\end{frame}




\section{The Transformation in \fsm}
\subsection{Design Idea}



\begin{frame}[fragile]
\frametitle{The Transformation in FSM}
%\framesubtitle{Design Idea}
\begin{scriptsize}
\begin{itemize}
\normalsize
\item<1->  \texttt{N = (make-ndfa S \sig{} s F \delt)}

\item<1->

Construction of the \texttt{dfa} in most textbooks:
\begin{description}
  \item [K$^{\prime}$ = ] 2$^K$
  \item[\sig{}$^{\prime}$ =] \sig{}
  \item[s$^{\prime}$ =] E(s)
  \item[F$^{\prime}$ =] \{B $|$ B$\subseteq$K$^{\prime}$ $\wedge$ B$\cap$F$\neq \emptyset$\}
  \item[\delt{}$^{\prime}$ =] \{(Q y $\bigcup$E(p)) $|$ p$\in$K $\wedge$ (q y p)$\in$\delt{} $\wedge$ Q$\in$K$^{\prime} \wedge$ q$\in$Q\} \newline
\end{description}
\end{itemize}
\end{scriptsize}
\end{frame}

\begin{frame}[fragile]
\frametitle{The Transformation in FSM}
%\framesubtitle{Design Idea}
\begin{scriptsize}
\begin{itemize}
\normalsize
\item<1->  \texttt{N = (make-ndfa S \sig{} s F \delt)} 

\item<1->What we do: find the transition function \arrow encode the \dfa{} 

\item<2-> Encoding of the \texttt{dfa}:
\begin{alltt}
 (make-dfa
   K\(\sp{\prime}\)=<encoding of super states>
   \sig{}
   s\(\sp{\prime}\)=<encoding of E(s)>
   F\(\sp{\prime}\)=<encoding of final super states>
   \delt{}\(\sp{\prime}\)=<encoded super states transition function>)
\end{alltt}
\end{itemize}
\end{scriptsize}
\end{frame}

\subsection{Illustrative Example}

\begin{frame}[fragile]
\frametitle{The Transformation in FSM}
%\framesubtitle{Illustrative Example}
\begin{scriptsize}

\begin{center}
\includegraphics[scale=0.5]{sample-ndfa.png}
\end{center}


\end{scriptsize}
\end{frame}


\begin{frame}[fragile]
\frametitle{The Transformation in FSM}
%\framesubtitle{Illustrative Example}
\begin{scriptsize}
\begin{itemize}
\normalsize
\item<1->
The result of computed empties for each state of the ndfa:
\begin{figure}[t!]
\centering
\begin{subfigure}{0.3\textwidth}
\begin{center}
\begin{tabular}{|c|l|}
  \hline
  % after \\: \hline or \cline{col1-col2} \cline{col3-col4} ...
  state & E(state) \\ \hline
  S & (S) \\ \hline
  A & (A) \\ \hline
  B & (B) \\ \hline
  C & (C) \\ \hline
  D & (D S) \\ \hline
  E & (E S) \\ \hline
\end{tabular}
\end{center}
\end{subfigure}
\hfill
\begin{subfigure}{0.5\textwidth}
\centering
\includegraphics[scale=0.35]{sample-ndfa.png}
\label{aa-ab}
\end{subfigure}
\end{figure}
\end{itemize}
\end{scriptsize}
\end{frame}

\begin{frame}[fragile]
\frametitle{The Transformation in FSM}
%\framesubtitle{Illustrative Example}
\begin{scriptsize}
\begin{itemize}
\normalsize
\item<1->
The following table summarizes the computed super state transition function: \newline
\begin{figure}[t!]
\centering
\begin{subfigure}{0.7\textwidth}
\begin{center}
\begin{tabular}{|l|l|l|}
  \hline
  % after \\: \hline or \cline{col1-col2} \cline{col3-col4} ...
  \texttt{Super State} & \texttt{a}         & \texttt{b} \\ \hline
  (S)         & (A B)     & () \\ \hline
  (A B)       & ()        & (C D S) \\ \hline
  (C D S)     & (E S A B) & () \\ \hline
  (E S A B)   & (A B)     & (C D S) \\ \hline
  ()          & ()        & () \\ \hline
\end{tabular}
\end{center}
\end{subfigure}
\hfill
\begin{subfigure}{0.4\textwidth}
\centering
\includegraphics[scale=0.35]{sample-ndfa.png}
\label{aa-ab}
\end{subfigure}
\end{figure}
\end{itemize}
\end{scriptsize}
\end{frame}


\begin{frame}[fragile]
\frametitle{The Transformation in FSM}
%\framesubtitle{Illustrative Example}
\begin{scriptsize}
\begin{itemize}
\normalsize
\item<1-> To build a dfa the super states must be mapped to \texttt{FSM} states. The following table is one such encoding:

\begin{center}
\begin{tabular}{|l|c|}
  \hline
  % after \\: \hline or \cline{col1-col2} \cline{col3-col4} ...
  \texttt{Super State} & \texttt{dfa} \ \texttt{State}  \\ \hline
  (S)         & S         \\ \hline
  (A B)       & A         \\ \hline
  (C D S)     & B         \\ \hline
  (E S A B)   & C         \\ \hline
  ()          & DEAD      \\ \hline
\end{tabular}
\end{center}
\end{itemize}
\end{scriptsize}
\end{frame}

\begin{frame}[fragile]
\frametitle{The Transformation in FSM}
%\framesubtitle{Illustrative Example}
\begin{scriptsize}
\begin{itemize}
\normalsize
\item<1->
\begin{figure}[t!]
\centering
\begin{subfigure}{0.7\textwidth}
\centering
\includegraphics[scale=0.35]{sample-ndfa.png}
\caption{\texttt{ndfa} transition diagram.}
\label{samplendfa-graph}
\end{subfigure}
\hfill
\item<2->
\begin{subfigure}{0.7\textwidth}
\centering
\includegraphics[scale=0.35]{sample-dfa.png}
\caption{\texttt{dfa} transition diagram.}
\label{sample-dfa}
\end{subfigure}
\end{figure}
\end{itemize}
\end{scriptsize}
\end{frame}

\section{Visualization in FSM}

\begin{frame}[fragile]
\frametitle{Visualization in FSM}
%\framesubtitle{Illustrative Example}
\begin{scriptsize}
\begin{itemize}
\normalsize
\item<1-> Two state diagrams built at every step.
\item<2->   N \arrow always fully displayed
\item<3->  D \arrow built one transition at a time.
\item<4-> We can step backwards!!! :)

\end{itemize}
\end{scriptsize}
\end{frame}


\begin{frame}[fragile]
\frametitle{Visualization in FSM}
%\framesubtitle{Illustrative Example}
\begin{scriptsize}
\begin{itemize}
\normalsize
\item<1-> WARNING: Names, names, names....

\end{itemize}
\end{scriptsize}
\end{frame}

\begin{frame}[fragile]
\frametitle{Visualization in FSM}
%\framesubtitle{Illustrative Example}
\begin{scriptsize}
\begin{itemize}
\normalsize
\item<1-> N's edges are partitioned into three color coded subsets: \newline

\begin{center}
\includegraphics[scale=0.7]{hedge.png}
\end{center}

\item<1-> HEDGES - \textcolor{magenta}{H}ighlighted \textcolor{magenta}{edges}

\end{itemize}
\end{scriptsize}
\end{frame}

\begin{frame}[fragile]
\frametitle{Visualization in FSM}
%\framesubtitle{Illustrative Example}
\begin{scriptsize}
\begin{itemize}
\normalsize
\item<1-> N's edges are partitioned into three color coded subsets: \newline

\begin{center}
\includegraphics[scale=0.7]{fedge.png}
\end{center}

\item<1-> FEDGES - \textcolor{magenta}{F}aded \textcolor{magenta}{edges}

\end{itemize}
\end{scriptsize}
\end{frame}

\begin{frame}[fragile]
\frametitle{Visualization in FSM}
%\framesubtitle{Illustrative Example}
\begin{scriptsize}
\begin{itemize}
\normalsize
\item<1-> N's edges are partitioned into three color coded subsets: \newline

\begin{center}
\includegraphics[scale=0.7]{bledge.png}
\end{center}

\item<1-> BLEDGES - \textcolor{magenta}{Bl}ack \textcolor{magenta}{edges}

\end{itemize}
\end{scriptsize}
\end{frame}

\begin{frame}[fragile]
\frametitle{Visualization in FSM}
%\framesubtitle{Illustrative Example}
\begin{scriptsize}
\begin{itemize}
\normalsize
\item<1-> N's edges are partitioned into three color coded subsets: \newline

\begin{center}
\includegraphics[scale=0.65]{FBH.png}
\end{center}

\end{itemize}
\end{scriptsize}
\end{frame}

\begin{frame}[fragile]
\frametitle{Visualization in FSM}
%\framesubtitle{Illustrative Example}
\begin{scriptsize}
\begin{itemize}

\item<1-> The visualization has to track processed and unprocessed edges of both the \ndfa{} and the \dfa{}. \newline


\begin{alltt}
\textcolor{magenta}{
;; a vst is a structure that consists of
;; upedges - unprocessed edges of the \dfa{}
;; pedges - processed edges of the \dfa{}
;; inodes - nodes included in the construction of the \dfa{}
;; M - given \ndfa{}
;; hedges - a list of highlighted edges in the ndfa graph
;; fedges - a list of faded edges in the ndfa graph
;; bledges - a list of black edges in the ndfa graph} \newline
(struct vst (upedges pedges inodes M hedges fedges bledges))
\end{alltt}



\end{itemize}
\end{scriptsize}
\end{frame}

\begin{frame}[fragile]
\frametitle{Visualization in FSM}
%\framesubtitle{Illustrative Example}
\begin{scriptsize}
\begin{itemize}
\normalsize

\item<1-> Moving the transformation one step forward: \newline



\noindent
\begin{minipage}[t]{0.48\linewidth}
\centering
\includegraphics[scale=0.31]{step1.png}
\end{minipage}
\hfill
\begin{minipage}[t]{0.48\linewidth}
\centering
\includegraphics[scale=0.3]{step2.png}
\end{minipage}




\end{itemize}
\end{scriptsize}
\end{frame}



\begin{frame}[fragile]
\frametitle{Visualization in FSM}
%\framesubtitle{Illustrative Example}
\begin{scriptsize}
\begin{itemize}
\normalsize

\item<1-> Moving the transformation one step backward: \newline

\noindent
\begin{minipage}[t]{0.48\linewidth}
\centering
\includegraphics[scale=0.3]{step2.png}
\end{minipage}
\hfill
\begin{minipage}[t]{0.48\linewidth}
\centering
\includegraphics[scale=0.31]{step1.png}
\end{minipage}


\end{itemize}
\end{scriptsize}
\end{frame}

\begin{frame}[fragile]
\frametitle{Visualization in FSM}
%\framesubtitle{Illustrative Example}
\begin{scriptsize}
\begin{itemize}

\item<1-> Completing the transformation in one step:

\noindent
\begin{minipage}[t]{0.48\linewidth}
\centering
\includegraphics[scale=0.31]{starting.png}
\end{minipage}
\hfill
\begin{minipage}[t]{0.48\linewidth}
\centering
\includegraphics[scale=0.28]{complete.png}
\end{minipage}


\end{itemize}
\end{scriptsize}
\end{frame}


\section{Concluding Remarks}

\begin{frame}[fragile]
\frametitle{Concluding Remarks}
\begin{scriptsize}
\begin{itemize}
\normalsize
\item<1-> Using this tool, any user can: \newline
\begin{itemize}
\item[\arrow]
Easily determine why the dfa’s super states exist \newline
\item[\arrow]
See how all ndfa edges are processed \newline
\item[\arrow]
Comprehend the development of the dfa’s transition function.

\end{itemize}

\item<2-> Unlike previous visualizations, our visualization moves \emph{backwards}.

\end{itemize}
\end{scriptsize}
\end{frame}

\begin{frame}[fragile]
\frametitle{Concluding Remarks}
\begin{scriptsize}
\begin{itemize}

\item<1-> Completed transformation in FSM and JFLAP. \newline


\begin{center}
\includegraphics[scale=0.3]{cmpr1.png}
\end{center}


\end{itemize}
\end{scriptsize}
\end{frame}

\begin{frame}[fragile]
\frametitle{Concluding Remarks}
\begin{scriptsize}
\begin{itemize}

\item<1-> Completed transformation in FSM and JFLAP. \newline


\begin{center}
\includegraphics[scale=0.3]{comparison.png}
\end{center}


\end{itemize}
\end{scriptsize}
\end{frame}

\begin{frame}[fragile]
\frametitle{Concluding Remarks}
\begin{scriptsize}
\begin{itemize}

\item<1-> Completed transformation in FSM and JFLAP. \newline


\begin{center}
\includegraphics[scale=0.3]{cmpr3.png}
\end{center}


\end{itemize}
\end{scriptsize}
\end{frame}

\begin{frame}[fragile]
\frametitle{Concluding Remarks}
\begin{scriptsize}
\begin{itemize}

\item<1-> Completed transformation in FSM and JFLAP. \newline


\begin{center}
\includegraphics[scale=0.3]{cmpr4.png}
\end{center}


\end{itemize}
\end{scriptsize}
\end{frame}

\begin{frame}[fragile]
\frametitle{Concluding Remarks}
\begin{scriptsize}
\begin{itemize}

\item<1-> Completed transformation in FSM and JFLAP. \newline


\begin{center}
\includegraphics[scale=0.3]{cmpr5.png}
\end{center}


\end{itemize}
\end{scriptsize}
\end{frame}

\begin{frame}[fragile]
\frametitle{Concluding Remarks}
\begin{scriptsize}
\begin{itemize}

\item<1-> Completed transformation in FSM and JFLAP. \newline


\begin{center}
\includegraphics[scale=0.3]{cmpr6.png}
\end{center}


\end{itemize}
\end{scriptsize}
\end{frame}


\begin{frame}[fragile]
\frametitle{Concluding Remarks}
\begin{scriptsize}
\begin{itemize}

\item<1-> Completed transformation in FSM and JFLAP. \newline


\begin{center}
\includegraphics[scale=0.3]{cmpr7.png}
\end{center}


\end{itemize}
\end{scriptsize}
\end{frame}

\begin{frame}[fragile]
\frametitle{Concluding Remarks}
\begin{scriptsize}
\begin{itemize}

\item<1-> Completed transformation in FSM and JFLAP. \newline


\begin{center}
\includegraphics[scale=0.3]{cmpr8.png}
\end{center}


\end{itemize}
\end{scriptsize}
\end{frame}

\begin{frame}[fragile]
\frametitle{Concluding Remarks}
\begin{scriptsize}
\begin{itemize}

\item<1-> Completed transformation in FSM and JFLAP. \newline


\begin{center}
\includegraphics[scale=0.3]{cmpr9.png}
\end{center}


\end{itemize}
\end{scriptsize}
\end{frame}

\begin{frame}[fragile]
\frametitle{Concluding Remarks}
\begin{scriptsize}
\begin{itemize}

\item<1-> Completed transformation in FSM and JFLAP. \newline


\begin{center}
\includegraphics[scale=0.3]{cmpr10.png}
\end{center}


\end{itemize}
\end{scriptsize}
\end{frame}

\begin{frame}[fragile]
\frametitle{Concluding Remarks}
\begin{scriptsize}
\begin{itemize}
\item<1-> Future work includes adding visualizations for other machine transformations or constructors. \newline
\begin{itemize}
\scriptsize
\item[\arrow] Transformations: ndfa\arrow regexp and regexp\arrow ndfa \newline
\item[\arrow] Constructors: closure properties under union, Kleene star, concatenation, complement, intersection
\end{itemize}


\end{itemize}
\end{scriptsize}
\end{frame}

\begin{frame}[fragile]
\frametitle{Concluding Remarks}
\begin{scriptsize}
\large
\begin{center}
\includegraphics[scale=0.17]{book-cover.jpg}
\end{center}

\begin{center}
Thank you for your attention! Any questions? :) \newline
\end{center}


\end{scriptsize}
\end{frame}


\end{document} 